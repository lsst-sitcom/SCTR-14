% generated from JIRA project LVV
% using template at /Library/Frameworks/Python.framework/Versions/3.11/lib/python3.11/site-packages/docsteady/templates/tpr.latex.jinja2.
% using docsteady version 3.0.0
% Please do not edit -- update information in Jira instead
\documentclass[PSE,lsstdraft,STR,toc]{lsstdoc}
\usepackage{geometry}
\usepackage{longtable,booktabs}
\usepackage{enumitem}
\usepackage{arydshln}
\usepackage{attachfile}
\usepackage{array}
\usepackage{dashrule}
\usepackage{pdfpages}

\newcolumntype{L}[1]{>{\raggedright\let\newline\\\arraybackslash\hspace{0pt}}p{#1}}

\input{meta.tex}

\newcommand{\attachmentsUrl}{https://github.com/\gitorg/\lsstDocType-\lsstDocNum/blob/\gitref/attachments}
\providecommand{\tightlist}{
  \setlength{\itemsep}{0pt}\setlength{\parskip}{0pt}}

\setcounter{tocdepth}{4}

\providecommand{\ul}[1]{\textbf{#1}}

\begin{document}

\def\milestoneName{Camera Hexapod Functional Re-Verification and Integration with SAL}
\def\milestoneId{}
\def\product{SIT-COM Integration}

\setDocCompact{true}

\title{LVV-P63: Camera Hexapod Functional Re-Verification and Integration with
SAL Test Plan and Report}
\setDocRef{\lsstDocType-\lsstDocNum}
\date{ 2025-03-03 }
\author{ Holger Drass }

% Most recent last
\setDocChangeRecord{
\addtohist{}{2019-12-06}{First Draft}{Austin Roberts}
}

\setDocCurator{Austin Roberts}
\setDocUpstreamLocation{\url{https://github.com/lsst-dm/\lsstDocType-\lsstDocNum}}
\setDocUpstreamVersion{\vcsrevision}



\setDocAbstract{
This is the test plan and report for
\textbf{ Camera Hexapod Functional Re-Verification and Integration with SAL},
an LSST milestone pertaining to the Project System Engineering and Commissioning.\\
This document is based on content automatically extracted from the Jira test database on \docDate.
The most recent change to the document repository was on \vcsDate.
}


\maketitle

\section{Introduction}
\label{sect:intro}


\subsection{Objectives}
\label{sect:objectives}

 The objective of this test plan is to re-verify the functional
requirements of the Camera Hexapod\textquotesingle s hardware and
software after shipment from the vendor\textquotesingle s facility to
the Summit, as defined in \citeds{LTS-206} and \citeds{LTS-160}.\\
This test campaign will only exercise the functionality that was
executed previously and meets the following criteria:

\begin{itemize}
\tightlist
\item
  Requires the vendor\textquotesingle s EUI software and hardware via
  local control
\item
  Requires control via SAL
\item
  Requires a laser tracker, mechanical gauges, temperature sensors
\item
  The first test cycle does \textbf{NOT} require the camera rotator to
  be loaded with the camera simulated mass or actual camera hardware
\item
  The second test cycle requires the camera rotator to be loaded with
  ComCam
\item
  Does not require the CCW or Camera Rotator to be operable.
\end{itemize}

\hfill\break
The hardware and software functional requirements were previously
verified during the test campaign by the vendor at the
vendor\textquotesingle s facility and accepted by LSST during the
Factory Acceptance Test review.



\subsection{System Overview}
\label{sect:systemoverview}

 The Camera Hexapod is mounted to the Camera Rotator with the primary
function of aligning the camera with the optical path of the telescope.


\subsection{Document Overview}
\label{sect:docoverview}

This document was generated from Jira, obtaining the relevant information from the
\href{https://rubinobs.atlassian.net/projects/LVV?selectedItem=com.atlassian.plugins.atlassian-connect-plugin:com.kanoah.test-manager__main-project-page\#\!/v2/testPlan/LVV-P63}{LVV-P63}
~Jira Test Plan and related Test Cycles (
\href{https://rubinobs.atlassian.net/projects/LVV?selectedItem=com.atlassian.plugins.atlassian-connect-plugin:com.kanoah.test-manager__main-project-page\#\!/testPlayer/LVV-R114}{LVV-R114}
\href{https://rubinobs.atlassian.net/projects/LVV?selectedItem=com.atlassian.plugins.atlassian-connect-plugin:com.kanoah.test-manager__main-project-page\#\!/testPlayer/LVV-R191}{LVV-R191}
).

Section \ref{sect:intro} provides an overview of the test campaign, the system under test (\product{}),
the applicable documentation, and explains how this document is organized.
Section \ref{sect:testplan} provides additional information about the test plan, like for example the configuration
used for this test or related documentation.
Section \ref{sect:personnel} describes the necessary roles and lists the individuals assigned to them.

Section \ref{sect:overview} provides a summary of the test results, including an overview in Table \ref{table:summary},
an overall assessment statement and suggestions for possible improvements.
Section \ref{sect:detailedtestresults} provides detailed results for each step in each test case.

The current status of test plan \href{https://rubinobs.atlassian.net/projects/LVV?selectedItem=com.atlassian.plugins.atlassian-connect-plugin:com.kanoah.test-manager__main-project-page\#\!/v2/testPlan/LVV-P63}{LVV-P63} in Jira is \textbf{ Approved }.

\subsection{References}
\label{sect:references}
\renewcommand{\refname}{}
\bibliography{lsst,refs,books,refs_ads,local}


\newpage
\section{Test Plan Details}
\label{sect:testplan}


\subsection{Data Collection}

  Observing is not required for this test campaign.

\subsection{Verification Environment}
\label{sect:hwconf}
  The Camera Hexapod will be verified in a climate-controlled environment
on the 3rd floor of the Summit Facility integrated with the Camera Cable
Wrap on the Camera Cart.

  \subsection{Entry Criteria}
  In order to test the Camera Hexapod functionality, the following
criteria must be met first:

\begin{itemize}
\tightlist
\item
  All the test setup for the Data Acquisition system must be completed
  and ready to record data for the laser tracker and inductive current
  probes
\item
  The Laser tracker and SMR\textquotesingle s are installed and setup
\item
  The Inductive current probes are installed and setup
\item
  All utilities and electrical connections are hooked up and allow the
  Camera Hexapod to be powered on and controlled
\item
  The EFD must be set up to be able to store events and telemetry data
\end{itemize}

  \subsection{Exit Criteria}
  In order for this event to be considered complete, the following
criteria must be met:

\begin{itemize}
\tightlist
\item
  Raw test data, events, and telemetry have been saved for the Camera
  Hexapod.
\item
  All test data has been analyzed and post-processed.
\item
  All test steps have been statused in the Jira Test Cases within this
  Test Plan, and actual results populated as required.
\item
  A summary of the results of the test campaign has been captured in the
  Overall Assessment and Recommended Improvements fields of this Test
  Plan.
\item
  A link to the verification artifacts used to produce the summary of
  results has been populated in the Verification Artifacts field of this
  Test Plan
\item
  Any failures have been captured in the
  \href{https://jira.lsstcorp.org/projects/FRACAS/issues/}{FRACAS}
  project.
\end{itemize}


\subsection{Related Documentation}

Docushare collection where additional relevant documentation can be found:

\begin{itemize}
\item None
\end{itemize}


\subsection{PMCS Activity}

Primavera milestones related to the test campaign:
See Epics in Traceability Tab


\newpage
\section{Personnel}
\label{sect:personnel}

The personnel involved in the test campaign is shown in the following table.

{\small
\begin{longtable}{p{3cm}p{3cm}p{3cm}p{6cm}}
\hline
\multicolumn{2}{r}{T. Plan \href{https://rubinobs.atlassian.net/projects/LVV?selectedItem=com.atlassian.plugins.atlassian-connect-plugin:com.kanoah.test-manager__main-project-page\#\!/v2/testPlan/LVV-P63}{LVV-P63} owner:} &
\multicolumn{2}{l}{\textbf{ Holger Drass } }\\\hline
\multicolumn{2}{r}{T. Cycle \href{https://rubinobs.atlassian.net/projects/LVV?selectedItem=com.atlassian.plugins.atlassian-connect-plugin:com.kanoah.test-manager__main-project-page\#\!/testPlayer/LVV-R114}{LVV-R114} owner:} &
\multicolumn{2}{l}{\textbf{
Holger Drass }
} \\\hline
\textbf{Test Cases} & \textbf{Assigned to} & \textbf{Executed by} & \textbf{Additional Test Personnel} \\ \hline
\multicolumn{2}{r}{T. Cycle \href{https://rubinobs.atlassian.net/projects/LVV?selectedItem=com.atlassian.plugins.atlassian-connect-plugin:com.kanoah.test-manager__main-project-page\#\!/testPlayer/LVV-R191}{LVV-R191} owner:} &
\multicolumn{2}{l}{\textbf{
Holger Drass }
} \\\hline
\textbf{Test Cases} & \textbf{Assigned to} & \textbf{Executed by} & \textbf{Additional Test Personnel} \\ \hline

\href{https://rubinobs.atlassian.net/projects/LVV?selectedItem=com.atlassian.plugins.atlassian-connect-plugin:com.kanoah.test-manager__main-project-page\#\!/v2/testCase/LVV-T1598}{LVV-T1598}
& {\small Holger Drass } & {\small Holger Drass } &
\begin{minipage}[]{6cm}
\smallskip
{\small (1) Optical Engineer (Laser Tracker Specialist)\\
(1) Systems Engineer }
\medskip
\end{minipage}
\\ \hline
\end{longtable}
}

\newpage

\section{Test Campaign Overview}
\label{sect:overview}

\subsection{Summary}
\label{sect:summarytable}

{\small
\begin{longtable}{p{2cm}cp{2.3cm}p{8.6cm}p{2.3cm}}
\toprule
\multicolumn{2}{r}{ T. Plan \href{https://rubinobs.atlassian.net/projects/LVV?selectedItem=com.atlassian.plugins.atlassian-connect-plugin:com.kanoah.test-manager__main-project-page\#\!/v2/testPlan/LVV-P63}{LVV-P63}:} &
\multicolumn{2}{p{10.9cm}}{\textbf{ Camera Hexapod Functional Re-Verification and Integration with SAL }} & Approved \\\hline
\multicolumn{2}{r}{ T. Cycle \href{https://rubinobs.atlassian.net/projects/LVV?selectedItem=com.atlassian.plugins.atlassian-connect-plugin:com.kanoah.test-manager__main-project-page\#\!/testPlayer/LVV-R114}{LVV-R114}:} &
\multicolumn{2}{p{10.9cm}}{\textbf{ Camera Hexapod Re-Verification }} & Done \\\hline
\textbf{Test Cases} &  \textbf{Ver.} & \textbf{Status} & \textbf{Comment} & \textbf{Issues} \\\toprule
 \multicolumn{2}{r}{ T. Cycle \href{https://rubinobs.atlassian.net/projects/LVV?selectedItem=com.atlassian.plugins.atlassian-connect-plugin:com.kanoah.test-manager__main-project-page\#\!/testPlayer/LVV-R191}{LVV-R191}:} &
\multicolumn{2}{p{10.9cm}}{\textbf{ Camera Hexapod Re-verification with ComCam }} & In Progress \\\hline
\textbf{Test Cases} &  \textbf{Ver.} & \textbf{Status} & \textbf{Comment} & \textbf{Issues} \\\toprule
\href{https://rubinobs.atlassian.net/projects/LVV?selectedItem=com.atlassian.plugins.atlassian-connect-plugin:com.kanoah.test-manager__main-project-page\#\!/v2/testCase/LVV-T1598}{LVV-T1598}
&
\\
 \hfill Execution & LVV-E1557
& Pass &
\begin{minipage}[]{9cm}
\smallskip
None
\medskip
\end{minipage}
&
                      \\\hline
     \caption{Test Campaign Summary}
\label{table:summary}
\end{longtable}
}

\subsection{Overall Assessment}
\label{sect:overallassessment}

\hfill\break
\textbf{The following results are for the first test-cycle
\href{https://jira.lsstcorp.org/secure/Tests.jspa\#/testCycle/LVV-C114}{LVV-C114}:}\\
The camera hexapod was tested at Level 3 without a payload attached.
Depending on the test case, the hexapod was controlled through the EUI
or the CSC.\\
The overall assessment of the first test case execution was: One test
case INITIAL PASS, and two test cases: FAIL.\\
After the third execution, all three test cases belonging to test
cycle~\href{https://jira.lsstcorp.org/secure/Tests.jspa\#/testCycle/LVV-C114}{LVV-C114}
of this test plan have the status: INITIAL PASS.\\
\strut \\
On the software side

\begin{itemize}
\tightlist
\item
  the EUI was reorganized to reflect the association of the commands
  with the states of the state machine correctly
  (\href{https://jira.lsstcorp.org/browse/DM-29738}{DM-29738})
\item
  the reaction of the state machine to the "clearError" command was
  improved (\href{https://jira.lsstcorp.org/browse/DM-29788}{DM-29788)}
\item
  some minor issues regarding rare events have easy workarounds.
  Therefore, the corresponding test steps passed with deviations.\\
  Specifically the inappropriate disconnection of the

  \begin{itemize}
  \tightlist
  \item
    encoder cable
    (\href{https://jira.lsstcorp.org/browse/DM-29791}{DM-29791})~
  \item
    power supplies
    (\href{https://jira.lsstcorp.org/browse/DM-29792}{DM-29792})~
  \item
    network connection
    (\href{https://jira.lsstcorp.org/browse/DM-29793}{DM-29793}).\\
    \strut \\
  \end{itemize}
\end{itemize}

The camera hexapod CSC software was improved by

\begin{itemize}
\tightlist
\item
  including checks into the camera hexapod CSC for position limits for
  the move and offset commands
  (\href{https://jira.lsstcorp.org/browse/DM-23092}{DM-23092)\\
  }
\item
  cleaning up XML for rotator and hexapods
  (\href{https://jira.lsstcorp.org/browse/DM-21699}{DM-21699})
\item
  Generating ~the inPosition event in the EFD
  (\href{https://jira.lsstcorp.org/browse/DM-29689}{DM-29689})
\item
  reporting the camera hexapod pivot point modifications in the EUI and
  the EFD (\href{https://jira.lsstcorp.org/browse/DM-29693}{DM-29693})
\item
  correcting the transition of the camera hexapod\textquotesingle s
  state machine back into standbyState
  (\href{https://jira.lsstcorp.org/browse/DM-29705}{DM-29705})
\item
  accepting the Disable command accepted and changing the state machine
  status correctly
  (\href{https://jira.lsstcorp.org/browse/DM-29706}{DM-29706})
\end{itemize}

\hfill\break
The configuration part of the CSC test case
(\href{https://jira.lsstcorp.org/secure/Tests.jspa\#/testCase/LVV-T1600}{LVV-T1600})
was not tested since the configuration system for all CSC is still under
development. The hexapod acceleration and velocity changes are not done
by stand-alone commands anymore and were, therefore, not tested. Since
the state machine of the CSC is still under development and will change
to a standbyState-entry state machine, the state machine transitions
were only partially tested as needed to conduct the other tests.\\
\strut \\
At the beginning of this verification activity, the hardware of the
camera hexapod presented issues in actuators 3 and 6. Both had dead
encoder zones that made the hexapod stop immediately. Recovery needed
direct intervention at the issue-causing actuator. The problem was
solved by taking out \textbf{all} actuators and servicing them by
cleaning the encoder band from Teflon and grease/oil and reassembling
the hexapod. In addition, during the service, the cables to the encoders
were found to be damaged due to the movements of the actuators. This
triggered a redesign of the actuator heads that is currently ongoing.
(see \href{https://jira.lsstcorp.org/browse/FRACAS-28}{FRACAS-28~}for
the actuator 3 failure
and~\href{https://jira.lsstcorp.org/browse/FRACAS-54}{FRACAS-54~}for the
actuator 6 failure).\\
\strut \\
Though not directly part of this verification testing, a random failure
(a feedback fault in Drive 0 and Drive 2) caused significant delays
during test execution. The encoders inside of the actuators as well as
the drives themselves, were first suspected as the cause for the
failure. The actuators could be excluded as the origin of the fault by
exchanging the cabling between the actuators. The drives did not present
any obvious failure reason. The issue was solved after finishing this
test by exchanging the cables to one actuator and testing that this
solution could be reproduced by changing the cable from between the
drives. All cables were exchanged. The problem did not appear again.\\
\strut \\
Apart from the aforementioned issues, the camera hexapod did not reach
the XYZ accuracy as required
(\href{https://jira.lsstcorp.org/browse/LVV-19501}{LVV-19501)~}at the
beginning of the testing campaign.\\
This hardware-related issue concerns the following requirements

\begin{itemize}
\tightlist
\item
  \href{https://jira.lsstcorp.org/browse/LVV-18622}{LTS-206-REQ-0164-V-02:
  3.5.12\_1 Positioning - LSST Re-verification}
  \href{https://jira.lsstcorp.org/browse/LVV-18631}{}
\item
  \href{https://jira.lsstcorp.org/browse/LVV-18631}{LTS-206-REQ-0178-V-02:
  3.5.24\_1 Hexapod Absolute Accuracy - LSST Re-verification}.
\end{itemize}

Specifically, the tests on the positioning in X, Y, and Z translation
combined with rotation failed to reach the required precision. The same
issue was observed for the hexapod\textquotesingle s absolute accuracy.
The reasons were lying, most likely in the measurement setup itself. The
laser tracker measurements are at the limit of the laser
tracker\textquotesingle s precision, and the MITUTOYO gauges mounts were
a preliminary solution. Testing the camera hexapod with an improved
MITUTOYO setup has shown that the same requirements are fulfilled for
the camera hexapod.\\
\strut \\
All executed hardware tests passed, as mentioned at the beginning of
this summary. Some measurements for the range in Z direction and
rotation around the Z-axis were given little priority since they are
testing a movement that is not expected to be used during normal
operations and were not performed due to missing time reasons.\\
\strut \\
\textbf{The following results are for the second
test-cycle~}\href{https://jira.lsstcorp.org/secure/Tests.jspa\#/testCycle/LVV-C191}{LVV-C191}:\\

\begin{itemize}
\tightlist
\item
  The main difference for this test cycle consists of ComCam being
  attached to the rotator.
\item
  All executed hardware tests passed again. Some measurements for the
  rotation around the Z-axis were given little priority since they are
  testing a movement that is not expected to be used during normal
  operations and were not performed due to missing time reasons.
\item
  The test case for the Camera Hexapod \textbf{Hardware
  Functional~}Re-verification
  (\href{https://jira.lsstcorp.org/secure/Tests.jspa\#/testCase/LVV-T1598}{LVV-T1598
  (1.0)}) has passed.
\item
  The
  \href{https://jira.lsstcorp.org/secure/Tests.jspa\#/testCase/LVV-T1599}{LVV-T1599}
  Camera Hexapod \textbf{Software Functional~}Re-verification test case
  was most successfully executed in the previous test cycle. Only one
  command was missing to be tested for the EUI. This test was
  successfully completed. Several requirements considering the telemetry
  and the Lookup Tables (LUTs) need to be tested together with the CSC
  (compensation mode) and were moved to the SAL test case LVV-T1600.
\end{itemize}

\subsection{Recommended Improvements}
\label{sect:recommendations}

\textbf{The following recommendations are for the situation after the
first test-cycle
\href{https://jira.lsstcorp.org/secure/Tests.jspa\#/testCycle/LVV-C114}{LVV-C114}:}

\begin{itemize}
\tightlist
\item
  To improve the situation before the next test cycle, it is recommended
  to finish the development of the camera hexapod low-level controller,
  the EUI, and the CSC.
\item
  The EUI and CSC software tests regarding the state machine should be
  executed when the state machine is updated to the standbyState-entry
  state machine. Beforehand, the test cases should be updated to account
  for state machine tests.
\item
  The configuration part of the CSC test case
  (\href{https://jira.lsstcorp.org/secure/Tests.jspa\#/testCase/LVV-T1600}{LVV-T1600})
  should be updated to account for the change in accordance with \citeds{LSE-209}
  and need to be tested for the first time.
\item
  The camera hexapod acceleration and velocity changes should now be
  tested as part of the configuration tests.
\item
  For the hardware tests, the camera hexapod should always be tested
  starting from the origin to avoid possible hysteresis.
\item
  Each test step to measure the absolute accuracy of the camera hexapod
  should be repeated at least three times to ensure the accuracy of the
  test results.
\item
  The following hardware tests from the MOOG testing sequence should be
  included

  \begin{itemize}
  \tightlist
  \item
    3.3.10 Hexapod Rotational Rz range
  \item
    3.3.1 ~Only the hexapod positioning in Rz. The rest was tested and
    is within specification.
  \item
    3.3.13 Measure Rz
  \end{itemize}
\end{itemize}

\hfill\break
\textbf{The following recommendations are for the situation after the
second test-cycle
\href{https://jira.lsstcorp.org/secure/Tests.jspa\#/testCycle/LVV-C191}{LVV-C191}:}\\

\begin{itemize}
\tightlist
\item
  The hardware-related tests need to be re-executed when the redesign
  and reconstruction of the actuator are finished.
\item
  This involves new hardware and is, therefore, not part of this test
  plan for re-verifying the vendor-delivered hardware.
\item
  During the test with the new hardware, including a test over the full
  range up to the software limits. It must include the +/- 8.7mm
  position.
\end{itemize}

\newpage
\section{Detailed Test Results}
\label{sect:detailedtestresults}

\subsection{Test Cycle LVV-R114 }

Open test cycle {\it \href{https://rubinobs.atlassian.net/projects/LVV?selectedItem=com.atlassian.plugins.atlassian-connect-plugin:com.kanoah.test-manager__main-project-page\#\!/testPlayer/LVV-R114}{Camera Hexapod Re-Verification}} in Jira.

Test Cycle name: Camera Hexapod Re-Verification\\
Status: Done

Re-verify the hardware and software requirements for the camera rotator
that MOOG previously tested.

\subsubsection{Software Version/Baseline}
\begin{enumerate}
\tightlist
\item
  Camera Hexapod Control Software with at least SAL v4.0
\item
  EFD with at least SAL v4.0
\end{enumerate}

\subsubsection{Configuration}
The configuration for the first test cycle is as follows:

\begin{itemize}
\tightlist
\item
  the hexapod is without a representative camera load
\item
  using the offlineState-entry state machine.
\end{itemize}

\subsubsection{Test Cases in LVV-R114 Test Cycle}

  %end of the if with theo test_items in testcycles_map[cyclie.id]

\subsection{Test Cycle LVV-R191 }

Open test cycle {\it \href{https://rubinobs.atlassian.net/projects/LVV?selectedItem=com.atlassian.plugins.atlassian-connect-plugin:com.kanoah.test-manager__main-project-page\#\!/testPlayer/LVV-R191}{Camera Hexapod Re-verification with ComCam}} in Jira.

Test Cycle name: Camera Hexapod Re-verification with ComCam\\
Status: In Progress

Re-verify the hardware and software requirements for the Camera Hexapod
that were previously tested by MOOG.

\subsubsection{Software Version/Baseline}
\begin{enumerate}
\tightlist
\item
  Camera Hexapod Control Software with at least SAL v5.0
\item
  EFD with at least SAL v5.0
\end{enumerate}

\subsubsection{Configuration}
The configuration for the second test cycle is:

\begin{itemize}
\tightlist
\item
  The hexapod is with ComCam installed
\item
  Using the standbyState-entry state machine.
\item
  Including the hardware configuration of the hexapod after the
  refurbishment of the actuators
\item
  Including the new cabling solving the random fault issues for Drive 0
  and Drive 2 (for details see
  \href{https://jira.lsstcorp.org/browse/FRACAS-64}{FRACAS-64} and
  \href{https://jira.lsstcorp.org/browse/FRACAS-56}{FRACAS-56})
\end{itemize}

\subsubsection{Test Cases in LVV-R191 Test Cycle}

\paragraph{ LVV-T1598 - Camera Hexapod Hardware Functional Re-Verification }\mbox{}\\

Version \textbf{1.0(d)}.
Status \textbf{Approved}.
Open  \href{https://rubinobs.atlassian.net/projects/LVV?selectedItem=com.atlassian.plugins.atlassian-connect-plugin:com.kanoah.test-manager__main-project-page\#\!/v2/testCase/LVV-T1598}{\textit{ LVV-T1598 } }
test case in Jira.

The objective of this test case is to re-verify the functional
requirements of the camera hexapod\textquotesingle s hardware after
shipment from the vendor\textquotesingle s facility to the Summit, as
defined in \citeds{LTS-206}.\\
This test case will only exercise the functionality that was executed
previously and meets the following criteria:

\begin{itemize}
\tightlist
\item
  It only requires the camera hexapod to be operable
\item
  Only requires the vendor\textquotesingle s EUI software and hardware
  via local control
\item
  Requires a laser tracker, mechanical gauges, induction current probe,
  temperature sensors
\item
  This test case can be executed with or without the camera rotator to
  be loaded with the camera simulated mass or actual camera hardware.\\
  \strut \\
\end{itemize}

The hardware functional requirements were previously verified during the
test campaign by the vendor at the vendor\textquotesingle s facility and
accepted by LSST during the Factory Acceptance Test review.\\
The test procedure used during the vendor\textquotesingle s acceptance
testing is the~\emph{LSST Hexapods-Rotator Acceptance Test Procedure}
which is attached to this test case.\\
The test steps of this test case reference the vendor\textquotesingle s
acceptance test procedure for the details on how to perform the test.\\
The reference to the vendor\textquotesingle s acceptance test procedure
is included to perform the test similarly as it was performed
previously.\\
There are also deviations to the vendor\textquotesingle s acceptance
test procedure included in the test cases.\\
This became necessary due to the differences in the verification
configuration and deviations to requirements granted to the vendor by
Rubin.\\
\strut \\
See the attached \emph{LSST Rotator Hexapod\textquotesingle s Manual}
for more information on how to operate the hexapod.

\textbf{ Preconditions}:\\ Prior to the execution of this test case to re-verify the Camera Hexapod
hardware functional requirements, the following Summit tasks must be
completed:

\begin{itemize}
\tightlist
\item
  The Hexapod has been installed on the camera cart

  \begin{itemize}
  \tightlist
  \item
    \url{https://jira.lsstcorp.org/browse/SUMMIT-3224}
  \end{itemize}
\item
  The Hexapod Controller has been deployed on the summit

  \begin{itemize}
  \tightlist
  \item
    \url{https://jira.lsstcorp.org/browse/SUMMIT-3229}
  \end{itemize}
\item
  Boxes for the Hexapod have been transported to the 3rd level

  \begin{itemize}
  \tightlist
  \item
    \url{https://jira.lsstcorp.org/browse/SUMMIT-3230}
  \end{itemize}
\item
  All Hexapod cables and cabinets have been prepared for integration
  with the camera cart

  \begin{itemize}
  \tightlist
  \item
    \url{https://jira.lsstcorp.org/browse/SUMMIT-3231}
  \end{itemize}
\item
  The offset has been installed onto the integrating structure

  \begin{itemize}
  \tightlist
  \item
    \url{https://jira.lsstcorp.org/browse/SUMMIT-3293}
  \end{itemize}
\item
  The Camera Hexapod electrical connections have been tested

  \begin{itemize}
  \tightlist
  \item
    \url{https://jira.lsstcorp.org/browse/SUMMIT-3294}
  \end{itemize}
\end{itemize}


Execution status: {\bf Pass }\\
Final comment:\\None



Detailed steps results LVV-R191-LVV-E1557 LVV-E1557-1243140632:\\
{\bf Note:} Steps "Not Executed" and with No Result are not shown in this report.
  %end of the if with theo test_items in testcycles_map[cyclie.id]


\newpage
\appendix
%Make sure lsst-texmf/bin/generateAcronyms.py is in your path
\section{Acronyms used in this document}\label{sec:acronyms}
\addtocounter{table}{-1}
\begin{longtable}{p{0.145\textwidth}p{0.8\textwidth}}\hline
\textbf{Acronym} & \textbf{Description}  \\\hline

EFD & Engineering and Facility Database \\\hline
GUI & Graphical User Interface \\\hline
LSST & Large Synoptic Survey Telescope \\\hline
LUT & Look-Up Table \\\hline
PMCS & Project Management Controls System \\\hline
Source & A single detection of an astrophysical object in an image, the characteristics for which are stored in the Source Catalog of the DRP database. The association of Sources that are non-moving lead to Objects; the association of moving Sources leads to Solar System Objects. (Note that in non-LSST usage "source" is often used for what LSST calls an Object.) \\\hline
\end{longtable}


\end{document}
